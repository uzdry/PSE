\documentclass[entwurf.tex]{subfiles}
\begin{document}

\section{Legende}
Vor allem in den Klassendiagrammen dieses Dokuments haben wir uns dazu entschlossen verschiedene Konventionen zur Übersicht und zum Verhindern von redundanten Informationen zu verwenden. Die selbst definierten sind in der folgenden Tabelle zu sehen.

\begin{tabularx}{\textwidth}{ l|X }
	Zeichen	& Beschreibung\\
	\hline
	. . .	& Eine Klasse mit nur drei Punkten ist in einem anderen Diagramm weiter erklärt.\\
	:Object	& z.B. als Parameter einer Funktion. Hier wäre der Name des Parameters praktisch der selbe wie der Typ.\\
	Getter/Setter & Bei Prototypen werden wegen Übersichtlichkeit offensichtliche Variablen, Getter/Setter und Funktionen wie z.B. getInstance bei Signleton nicht extra dazugeschrieben
\end{tabularx}

\end{document}

