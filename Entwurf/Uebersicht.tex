\documentclass[entwurf.tex]{subfiles}
\begin{document}

\section{Einleitung}
Dieses Dokument beschreibt den Entwurf des Projekts VINJAB (VINJAB Is Not Just A Boardcomputer). Es beinhaltet den Aufbau der Software, welche Bibliotheken und Frameworks wo eingesetzt werden, sowie die gesamte Klassenstruktur.

\section{Legende}
Vor allem in den Klassendiagrammen dieses Dokuments haben wir uns dazu entschlossen, verschiedene Konventionen zur Verbesserung der Übersicht und zum Vermeiden von Redundanzen zu benutzen. Konventionen, die nicht im UML-Standard enthalten sind, sind in der folgenden Tabelle erläutert.

\begin{tabularx}{\textwidth}{ l|X }
	Zeichen	& Beschreibung\\
	\hline
	. . .	& Eine Klasse, die nur drei Punkte enthält, ist in einem anderen Diagramm beschrieben und wird an dieser Stelle nur referenziert.\\
	:Object	& z.B. als Parameter einer Funktion. Der Name dieses Parameters ist identisch mit seinem Datentyp.\\
	Getter/Setter & Aus Gründen der Übersicht werden offensichtliche Variablen, Getter, Setter und Entwurfsmusterspezifische Funktionen z.T.  nicht explizit erwähnt.
\end{tabularx}

\end{document}

