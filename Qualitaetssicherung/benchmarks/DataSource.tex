\documentclass[qualitaetssicherung.tex]{subfiles}

\begin{document}

\section{Data Source}
Die Geschwindigkeit der Aktualisierung der Daten über Auto hängt von dem CAN-Bus
im Auto, dem Bluetooth-Adapter, und der Frequenz des Abschicken der Befehle ab.
Um alle Befehle eine sinnvolle Antwort zu haben, hat das Programm ein
Mechanismus genutzt, dass jeder Befehl erst dann abgeschickt werden kann, wenn der
Antwort des letzten Befehles ankommt. Durchschnittlich ist es 80ms zwischen zwei
Befehle. Das heißt, pro Sekunde werden durchschnittlich 12.5 Befehle erforgreich
nach OBD geschickt.

\subsection{Server-Side Bus}
	Der Bus auf der Serverseite arbeitet synchrone Methodenaufrufe (\code{handleMessage(m: Message): void}) von allen \code{BusDevice} Instanzen ab, die auf das \code{Topic} von \code{m} subscriben. Man kann kontinuierlich Nachrichten auf den Bus legen. Auf den Referenzgeräten wurden dadurch circa 2/3 der CPU leistung in Anspruch genommen.

\end{document}