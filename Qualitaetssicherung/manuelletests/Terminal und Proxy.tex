\documentclass[qualitaetssicherung.tex]{subfiles}

\begin{document}

\section{Terminal und Proxy testen}

	\subsection{Server und Client verbinden}
		\begin{itemize}
			\item
			Es ist die Verbindungaufbau zwischen den Server und Endgerät zu testen, indem
			man durch ein Browser den Server anfragt und schaut, ob die
			Startkonfiguration erfolgreich angezeigt werden kann.
			\item
			Vorbedingung: Der Server ist Startet. Ein Endgerät ist durch WLAN mit der
			Server verbunden.
			\item
			Nachbedingung (Erfolg): Die Startkonfiguration der Dashboards werden im
			Browser angezeigt.
			\item
			Nachbedingung (Fehlschlag): Im Browser wird die Startkonfiguration nicht
			richtig angezeigt.
			\item
			Beschreibung/Testschritte:
			\begin{itemize}
				\item
				Schritt eins: Der Benutzer startet den Server im Rasberry-PI.
				\item
				Schritt zwei: Der Benutzer verbindet sein Endgerät durch WLAN mit dem
				Server.
				\item  
				Schritt drei: Der Benutzer öffnet den Browser und geht zur Addresse des
				Servers.
			\end{itemize}
		\end{itemize}
		
	\subsection{Multibenutzer}
		\begin{itemize}
			\item
			Es ist die Fähigkeit, dass die Software mit mehreren Benutzern laufen kann.
			\item
			Vorbedingung: Der Server ist Startet und kann die Daten über ein Auto
			einsammlen durch OBD-Bluetooth-Adapter. Mehrere Endgeräte sind durch WLAN
			mit der Server verbunden. Der Fahrer fährt mit dem Auto.
			\item
			Nachbedingung (Erfolg): Die Endgeräte können die echtzeitige Daten über das
			Auto nach Wunsch anzeigen.
			\item
			Nachbedingung (Fehlschlag): Die Endgeräte können die Daten nicht richtig
			anzeigen.
			\item
			Beschreibung/Testschritte:
			\begin{itemize}
				\item
				Schritt eins: Der Fahrer steckt das OBD-Bluetooth-Adapter in das Auto
				ein und lässt das Auto anzünden.
				\item
				Schritt zwei: Der Fahrer startet den Server auf dem Rasberry PI.
				\item
				Schritt zwei: Der Benutzer verbindet seine Endgeräte mit dem Server.
				\item 
				Schritt Drei: Der Benutzer öffnet die Browser in verschiedenen Geräte
				und geht die Addresse zur Server.
				\item 
				Schritt Vier: Der Benutzer fordert unterschiedliche Widgets in
				Endgeräte und überprüfen, ob die angekommenen Daten mit Realität
				übereinstimmen.
			\end{itemize}
		\end{itemize}
\end{document}