\documentclass[qualitaetssicherung.tex]{subfiles}

\begin{document}

\section{Terminal und Proxy testen}

	\subsection{Server und Client verbinden}
		\begin{itemize}
			\item
			Es ist die Verbindungaufbau zwischen den Server und Endgerät zu testen, indem
			man durch ein Browser den Server anfragt und schaut, ob die
			Startkonfiguration erfolgreich angezeigt werden kann.
			\item
			Vorbedingung: Der Server ist gestartet. Ein Endgerät ist durch WLAN mit dem
			Server verbunden.
			\item
			Nachbedingung (Erfolg): Die Startkonfiguration der Dashboards werden im
			Browser angezeigt.
			\item
			Nachbedingung (Fehlschlag): Im Browser wird die Startkonfiguration nicht
			richtig angezeigt.
			\item
			Beschreibung/Testschritte:
			\begin{itemize}
				\item
				Schritt eins: Der Benutzer startet den Server im Raspberry-PI.
				\item
				Schritt zwei: Der Benutzer verbindet sein Endgerät durch WLAN mit dem
				Server.
				\item  
				Schritt drei: Der Benutzer öffnet den Browser und geht zur Adresse des
				Servers.
			\end{itemize}
		\end{itemize}
		
	\subsection{Multibenutzer}
		\begin{itemize}
			\item
			Es ist die Fähigkeit, dass die Software mit mehreren Benutzern laufen kann.
			\item
			Vorbedingung: Der Server ist gestartet und kann die Daten über ein Auto
			einsammeln durch OBD-Bluetooth-Adapter. Mehrere Endgeräte sind durch WLAN
			mit dem Server verbunden. Der Fahrer fährt mit dem Auto.
			\item
			Nachbedingung (Erfolg): Die Endgeräte können die Echtzeitdaten über das
			Auto nach Wunsch anzeigen.
			\item
			Nachbedingung (Fehlschlag): Die Endgeräte können die Daten nicht richtig
			anzeigen.
			\item
			Beschreibung/Testschritte:
			\begin{itemize}
				\item
				Schritt eins: Der Fahrer steckt das OBD-Bluetooth-Adapter in das Auto
				ein und lässt das Auto anzünden.
				\item
				Schritt zwei: Der Fahrer startet den Server auf dem Raspberry PI.
				\item
				Schritt drei: Der Benutzer verbindet seine Endgeräte mit dem Server.
				\item 
				Schritt vier: Der Benutzer öffnet den Browser auf verschiedenen Geräten
				und geht zur Adresse des Servers.
				\item 
				Schritt fünf: Der Benutzer fordert unterschiedliche Widgets auf den
				Endgeräten und überprüft, ob die angekommene Daten mit der Realität
				übereinstimmen.
			\end{itemize}
		\end{itemize}
			\subsection{Client Value Subscriptions}
		\begin{itemize}
			\item Lässt sich der User einen bislang nicht gezeigten Wert anzeigen, wird dieser auch auf Server Seite (vom Proxy des entsprechenden Clients) subscribed.
			\item Wenn der Client die Verbindung trennt, soll der Proxy auf Server-Seite gelöscht werden. Davor sollen aber alle Subscriptions des Proxy gelöscht werden. Die Methode \code{unsubscribeAll(): void} der Klasse Bus wird in den Unit-Tests abgedeckt.  ~\ref{}
			\item Vorbedingung: Es muss mindestens ein Client mit dem Server verbunden sein und mindestens einen Wert anzeigen.
			\item
			Nachbedingung (Erfolg): Für kein \code{Topic topic} ist Der Proxy im Array \code{Broker.get().subscribers[topic]} enthalten
			\item
			Nachbedingung (Fehlschlag): in der Hashmap \code{Broker.get().subscribers} taucht eine Referenz auf einen Proxy auf, der keine Kommunikation zu einem Client hat.
		\end{itemize}
\end{document}