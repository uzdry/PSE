\documentclass[qualitaetssicherung.tex]{subfiles}

\begin{document}

\section{UI}

	\subsection{Allgemeine Widget-Funktion}
		\begin{itemize}
			\item
			Teste das Erzeugen von verschiedenen Widgets auf dem Dashboard. Der Test wird über den localhost also nicht über den Raspberry Pi durchgeführt, da der komplette Test nur auf dem Client läuft und der Pi dem Browser dieselben Dateien anbietet.
			\item
			Vorbedingung: Der Server muss laufen.(Optional: Internetverbingung für GoogleMaps)
			\item
			Nachbedingung (Erfolg): Alle Widgets lassen sich mit verschiedenen Signalen erzeugen. Ihre Größe ist änderbar. Sie lassen sich verschieben und auch löschen. 
			\item
			Nachbedingung (Fehlschlag): Ein Widget lässt sich nicht erzeugen, verschieben, oder transformieren. Oder eine Fehlermeldung wird geworfen.
			\item
			Beschreibung/Testschritte:
			\begin{itemize}
				\item
				Öffne im Browser 10.10.0.1:8000
				\item
				Wechsle in den Edit-Modus über den passenden Button.
				\item
				Wähle zufällig ein Signal über das obere Dropdown Menü
				\item
				Wähle zufällig ein Widget im unteren Dropdown Menü
				\item
				Klicke auf den "Add"-Button
				\item
				Verschiebe das Widget an eine zufällige Stelle
				\item
				Wiederhole die letzten 4 Schritte einige male
				\item
				Zuletzt aktiviere den Löschen-Modus über die Checkbox
				\item
				Lösche alle erzeugten Widgets
			\end{itemize}
		\end{itemize}
		
	\subsection{Beispiel - Testname 2}
		\begin{itemize}
			\item
			Eine kurze Zusammenfassung. Was wird getestet, warum... usw.
			\item
			Vorbedingung: (Beispiel: manche UI Elemente müssen schon geöffnet sein)
			\item
			Nachbedingung (Erfolg): (Beispiel: dies und das wird angezeigt)
			\item
			Nachbedingung (Fehlschlag): (Beispiel: wird nicht angezeigt, stürzt ab...)
			\item
			Beschreibung/Testschritte:
			\begin{itemize}
				\item
				Schritt eins...
				\item
				Schritt zwei
			\end{itemize}
		\end{itemize}
\end{document}