\documentclass[qualitaetssicherung.tex]{subfiles}

\begin{document}

\section{UI}

	

	\subsection{Allgemeine Widget-Funktion}
		\begin{itemize}
			\item
			Teste das Erzeugen von verschiedenen Widgets auf dem Dashboard. Der Test wird über den localhost also nicht über den Raspberry Pi durchgeführt, da der komplette Test nur auf dem Client läuft und der Pi dem Browser dieselben Dateien anbietet.
			\item
			Vorbedingung: Der Server muss laufen.(Optional: Internetverbingung für GoogleMaps)
			\item
			Nachbedingung (Erfolg): Alle Widgets lassen sich mit verschiedenen Signalen erzeugen. Ihre Größe ist änderbar. Sie lassen sich verschieben und auch löschen. 
			\item
			Nachbedingung (Fehlschlag): Ein Widget lässt sich nicht erzeugen, verschieben, oder transformieren. Oder eine Fehlermeldung wird geworfen.
			\item
			Beschreibung/Testschritte:
			\begin{itemize}
				\item
				Öffne im Browser 10.10.0.1:8000
				\item
				Wechsle in den Edit-Modus über den passenden Button.
				\item
				Wähle zufällig ein Signal über das obere Dropdown Menü
				\item
				Wähle zufällig ein Widget im unteren Dropdown Menü
				\item
				Klicke auf den "Add"-Button
				\item
				Verschiebe das Widget an eine zufällige Stelle
				\item
				Wiederhole die letzten 4 Schritte einige male
				\item
				Zuletzt aktiviere den Löschen-Modus über die Checkbox
				\item
				Lösche alle erzeugten Widgets
			\end{itemize}
		\end{itemize}
		
	\subsection{Dashboard-Konfiguration}
		\begin{itemize}
			\item
			Man testet ob die Dashboard-Konfiguration gespeichert wird. Und ob man diese wieder aufrufen kann.
			\item
			Vorbedingung: Der Server muss laufen (Optional: Internetverbindung für Google Maps)
			\item
			Nachbedingung (Erfolg): Die gewählte Konfiguration wird wieder angezeigt.
			\item
			Nachbedingung (Fehlschlag): Eine andere oder keine Konfiguration wird angezeigt
			\item
			Beschreibung/Testschritte:
			\begin{itemize}
				\item
				Öffne im Browser 10.10.0.1:8000
				\item
				Wechsle in den Edit-Modus über den passenden Button.
				\item
				Wähle zufällig ein Signal über das obere Dropdown Menü
				\item
				Wähle zufällig ein Widget im unteren Dropdown Menü
				\item
				Klicke auf den "Add"-Button
				\item
				Verschiebe das Widget an eine zufällige Stelle
				\item
				Wiederhole die letzten 4 Schritte beliebige male
				\item 
				Wechsle wieder in den Dashboard-Modus
				\item
				Lade die Homepage neu
			\end{itemize}
		\end{itemize}
		
	\subsection{Replay Modus}
		\begin{itemize}
			\item
			Man testet ob der Replay-Modus funktioniert. Also ob man auf Befehl eine vergangene Fahrt anzeigen lassen kann.
			\item
			Vorbedingung: Der Server muss laufen (Optional: Internetverbindung für Google Maps) und er muss bereits mindestens eine Fahrt hinter sich haben.
			\item
			Nachbedingung (Erfolg): Nach spätestens ein paar Sekunden werden Daten einer vergangenen Fahrt angezeigt.
			\item
			Nachbedingung (Fehlschlag): Es werden entweder keine neuen Daten mehr angezeigt oder weiterhin die aktuellen Live-Daten.
			\item
			Beschreibung/Testschritte:
			\begin{itemize}
				\item
				Öffne im Browser 10.10.0.1:8000
				\item
				Wechsle in den Replay-Modus über den passenden Button.
				\item
				Wähle zufällig eine Fahrt über das Dropdown Menü
				\item
				Klicke auf den "Start"-Button
			\end{itemize}
		\end{itemize}	
\end{document}