% Version 0.1
\documentclass[a4paper,oneside,10pt]{report}

\usepackage{subfiles}
\usepackage[ngerman]{babel}
\usepackage[T1]{fontenc}
\usepackage[utf8]{inputenc}
\usepackage{graphicx}
\usepackage[cc]{titlepic}
\usepackage{float}

\def\code#1{\texttt{#1}}

\begin{document}

\pagestyle{empty}


\title{PSE - VINJAB: Resultate der Qualitätssicherung}
\author{Jonas Haas 
			\and Nicolas Schreiber
			\and Valentin Springsklee
			\and Yimeng Zhu
			\and David Grajzel}
\titlepic{\includegraphics[width=\textwidth]{Images/VINJAB-Logo.png}}
\maketitle

%% Inhaltsverzeichnis %%%%%%%%%%%%%%%%%%%%%%%%%%%%%%%%%%%%%%%
\tableofcontents %Inhaltsverzeichnis
\cleardoublepage %Das erste Kapitel soll auf einer ungeraden Seite beginnen.

\pagestyle{plain} %%Ab hier die Kopf-/Fusszeilen: headings / fancy / ...



\chapter{Einführung}\label{einfuehrung}

\subfile{einfuehrung/einfuehrung}
\newpage


\chapter{Statische Code-Analyse}\label{statischeanalyse}

\subfile{statischeanalyse/zusammenfassung}
\newpage


\chapter{Unit Tests}\label{unittests}

\subfile{unittests/bus}
\newpage
\subfile{unittests/BluetoothOBD}
\newpage
\subfile{unittests/ui}
\newpage
\subfile{unittests/rueckfahrkamera}
\newpage
\subfile{unittests/settings}
\newpage


\chapter{Manuelle Tests}\label{manuelletests}

\subfile{manuelletests/ui}

\subfile{manuelletests/Terminal-Proxy}

\newpage


\chapter{Testszenarien}\label{testszenarien}

%% Klar: die Testszenarien, die wir durchführen, werden sinnvoller benannt...
\subfile{testszenarien/BluetoothOBD}
\newpage


\chapter{Benchmarks}\label{benchmarks}

\subfile{benchmarks/firefox_svg}

\subfile{benchmarks/settings_speedtest}

\subfile{benchmarks/DataSource}
\newpage


%% Abbildungsverzeichnis
\clearpage
\addcontentsline{toc}{chapter}{Abbildungsverzeichnis}
\listoffigures

\appendix

\end{document}