\documentclass[qualitaetssicherung.tex]{subfiles}

\begin{document}

\section{Rückfahrkamera}

	\subsection{RGB Objekte dürfen nur mit drei bytes initialisiert werden.}
		\begin{itemize}
			\item
			Symptom: Es war möglich RGB Objekte mit r/g/b Werten außerhalb [0,255] zu initialisieren.
			\item
			Grund: Es gab keinen min/max Limiter im Constructor.
			\item
			Behebung: Constructor angepasst, jetzt werden alle Werte auf [0,255] abgebildet (truncated).
		\end{itemize}
		
	\subsection{RGB Objekte sind doch keine RGG Objekte.}
		\begin{itemize}
			\item
			Symptom: Man initialisierte RGB mit r/g/b und bei Nachfrage bekam die Werte r/g/g zurück.
			\item
			Grund: Constructor hat this.blue auf Math.max(0, this.green) gesetzt.
			\item
			Behebung: Tippfehler korrigiert.
		\end{itemize}
	\subsection{Geometry.Vec4 initialisierung mit null war fehlerhaft}
		\begin{itemize}
			\item
			Symptom: Man initialisierte Geometry.Vec4 mit NULL und getValues gab NULL zurück, statt (0,0,0,0).
			\item
			Grund: Reihenfolge der Code-Zeilen im Konstruktor vertauscht.
			\item
			Behebung: Reihenfolge der Zeilen korrigiert.
		\end{itemize}
	\subsection{Geometry.Vec4 Zugriff auf null trotz Gegenmaßnahmen}
		\begin{itemize}
			\item
			Symptom: Zugriff auf Attributwert null, obwohl if (!null == true).
			\item
			Grund: Zwei if() statements nacheinander sahen folgt aus: if(null), if(…). Man sollte aber an dieser Stelle if(null), else if(…) verwenden.
			\item
			Behebung: If auf else if geändert.
		\end{itemize}
	\subsection{ArrayNormalizer verkürzt Arrays nicht}
		\begin{itemize}
			\item
			Symptom: Array der Länge 3 wird eingegeben, ArrayNormalizer soll dies auf Länge 2 normalisieren, die Ausgabe hat trotzdem die Länge 3.
			\item
			Grund: Fallunterscheidung fehlt, Fall 1: Ziellänge < Eingabelänge (in diesem Fall soll der Array verlängert werden), Fall 2: Ziellänge > Eingabelänge (hier soll der Array gekürzt werden).
			\item
			Behebung: Fallunterscheidung hinzugefügt.
		\end{itemize}
		
		\subsection*{Resultate / Statistiken}
		Hier sind die Testresultate zu sehen. Es ist dabei zu beachten, dass GUI-Klassen und solche Adapter-Klassen, die eine komplexe Umgebung/Initialisierung brauchen evtl. nicht (vollständig) abgedeckt sind.
			\begin{center}
					\begin{tabular}{| l | l | l | l | l |}
					\hline
					Dateiname & Abdeckung & Jasmine Expects & Code LOC & Spec LOC\\ \hline
					Format & 100.00\% & 24 & 160 & 162\\ \hline
					Geometry & 100.00\% & 265 & 397 & 612 \\ \hline
					Primitive & 100.00\% & 61 & 107 & 137\\ \hline
					\hline
					\end{tabular}
			\end{center}

\end{document}