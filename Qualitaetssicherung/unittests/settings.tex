\documentclass[qualitaetssicherung.tex]{subfiles}

\begin{document}

\section{Settings}

	\subsection{Settings directory append child/set parent Problem}
		\begin{itemize}
			\item
			Symptom: The set parent Funktion arbeitete nicht, wie gewünscht, beim Elternknoten wurde das Kind nicht hinzugefügt, wenn die Funktion set parent beim Kindknoten aufgerufen wurde. 
			\item
			Grund: Fehlerhafte Implementierung, kein Aufruf der Funktion append child des Elternknotens.
			\item
			Behebung: Jetztige auf eine funktionierende Implementierung geändert.
		\end{itemize}
	
	\subsection{Settings (L/N) Parameter mit Client side buffer == null}
		\begin{itemize}
			\item
			Symptom: Settings List und Numeric (L/N) Parameter SetValue Funktion stürzt ab, wenn die Klasse SettingsLParameter mit ClientSideBuffer == null initialisiert wurde.
			\item
			Grund: Es wird nicht im Constructor und auch nicht in der Funktion geprüft, ob ClientSideBuffer undeclared oder null ist.
			\item
			Behebung: Codezeile, die prüft ob ClientSideBuffer null ist wurde hinzugefügt.
		\end{itemize}
	
	\subsection{Settings List Parameter MVC View Aktualisierung}
		\begin{itemize}
			\item
			Symptom: Wenn man setValue() in Settings List Parameter (Controller) aufruft, dann soll der Wert in View und Model entsprechend geändert werden. Im Model erfolgt die Änderung, die Anzeige ändert sich jedoch nicht.
			\item
			Grund: Code-Zeile, die bei einer Wertänderung das Modell auch ändert ist nicht vorhanden.
			\item
			Behebung: Fehlende Code-Zeile wurde hinzugefügt.
		\end{itemize}
		
	\subsection{Settings Parameter MVC View Index}
		\begin{itemize}
			\item
			Symptom: Wenn man beim Controller das letzte Element auswählt, dann soll es in View auch ausgewählt werden. Liest man jedoch selectedIndex zurück, bekommt man den Wert -1, statt zum Beispiel 5, wenn die Liste 5 Elemente hat. 
			\item
			Grund: Index fängt in View bei 0 an.
			\item
			Behebung: Die Zeile, die den Wert von selectedIndex auf n setzt wurde geändert, so wird jetzt den Wert auf n-1 gesetzt. Hinweis: die Liste der Elemente fängt in XML Dateien weiterhin bei 1 an.
		\end{itemize}
		
	\subsection{Settings Parameter get Full Uid wenn Parent == null}
		\begin{itemize}
			\item
			Symptom: Funktion getFullUid() in SettingsParameter stürzt beim Aufruf ab, wenn SettingsParameter mit Parent == null initialisiert wurde, also wenn setParent() nicht früher aufgerufen wird.
			\item
			Grund: Die Funktion greift auf this.parent zu, was in diesem Fall noch nicht auf einen Wert != null gesetzt wurde, weil man früher die Funktion setParent() nicht aufgerufen hat.
			\item
			Behebung: Es wird in getFullUid() geprüft ob man den Elternknoten schon auf einen gültigen Wert gesetzt hat. Wenn nicht dann wird 
		\end{itemize}
		
	\subsection{Settings Parameter Elternknoten ist nicht definiert}
		\begin{itemize}
			\item
			Symptom: Funktion getParent() wird aufgerufen und man bekommt undefined statt null zurück, wenn man früher den Elternknoten nicht explizit definiert hat.
			\item
			Grund: Es gibt keinen Initialwert von this.parent, solange es nicht explizit auf einen Wert (auch null) gesetzt wird, ist es undefined.
			\item
			Behebung: This.parent wird immer mit null initialisiert.
		\end{itemize}
		
	\subsection{Settings Numeric Parameter Min/Max Limiter}
		\begin{itemize}
			\item
			Symptom: Wenn der Initialwert außerhalb von [Min, Max] liegt, wird es auf [Min, Max] abgebildet. Setzt man jedoch mit setValue() den Wert, werden die Mindestwert/Maximalwert Attribute nicht berücksichtigt und man kann den Attribut Wert beliebig setzen.
			\item
			Grund: Es wird in setValue() nicht geprüft, ob der Wert, den man jetzt eingibt gültig ist (ob es innerhalb [Min, Max] liegt.
			\item
			Behebung: Entsprechende Code-Zeilen addiert, wenn der Eingabewert ausserhalb [Min, Max] liegt, wird es auf Min (falls Wert < Min) oder Max (falls Wert > Max) gesetzt.
		\end{itemize}
		
\subsection*{Resultate / Statistiken}
		Hier sind die Testresultate zu sehen. Es ist dabei zu beachten, dass GUI-Klassen und solche Adapter-Klassen, die eine komplexe Umgebung/Initialisierung brauchen evtl. nicht (vollständig) abgedeckt sind.
\begin{center}
    \begin{tabular}{| l | l | l | l | l |}
    \hline
    Dateiname & Abdeckung & Jasmine Expects & Code LOC & Spec LOC\\ \hline
    CompositeStructure & 94.58\% & 58 & 612 & 165 \\ \hline
		SettingsData & 99.07\% & 58 & 188 & 252 \\
    \hline
    \end{tabular}
\end{center}
	
\end{document}