\documentclass[pflichtenheft.tex]{subfiles}

\begin{document}

\chapter{Globale Testfälle}

\renewcommand{\theenumi}{/GT\ifnum \value{enumi}<10 0\fi\arabic{enumi}0/}
\renewcommand{\labelenumi}{\theenumi}
\renewcommand{\theenumii}{\arabic{enumii}}
\renewcommand{\labelenumii}{GT\ifnum \value{enumi}<10 0\fi\arabic{enumi}\arabic{enumii}/}

\textbf{globale Vorbedingungen:}
\begin{itemize}
\item
Die Zündung ist an.
\item
Der Server läuft.
\end{itemize}

\begin{enumerate}

\item{\textbf{Instrumente anzeigen, ausblenden und Dashboard überladen}} \\
\textbf{Vorbedingung: } Ein Endgerät ist mit dem Server verbunden. \\
\textbf{Ablauf: } Alle verfügbaren Instrumente auf dem Endgerät Einblenden, Ausblenden und Kombinieren

\item{\textbf{Protokoll führen und Daten anzeigen}} \\
\textbf{Vorbedingung: } Das Auto fährt. \\
\textbf{Ablauf: } Mit dem Auto fahren und Daten erzeugen. Die Daten im Endgerät auf allen Instrumenten live und auf Abruf anzeigen. Auf Konsistenz und ausreichend kleine Latenz prüfen. Datenausfälle simulieren. Datenbank auf korrekte Einträge überprüfen.

\item{\textbf{Maximale Kapazität prüfen} \\
\textbf{Vorbedingung: } keine weiteren Vorbedingungen erforderlich \\
\textbf{Ablauf: } Die default-Kapazität auf dem Endgerät prüfen, neue default-Kapazität setzen. Daten mit dem Auto erzeugen, bis die Datenbank bis zum Grenzwert gefüllt ist. Überprüfen, ob das Datenlimit eingehalten wurde und ob nur wie spezifiziert die ältesten Daten gelöscht wurden. Auf Konsistenz und korrektes Löschen aller Daten prüfen.

\item{\textbf{Remote Access}} \\
\textbf{Vorbedingung: } keine weiteren Vorbedingungen erforderlich \\
\textbf{Ablauf: } Mit entferntem Endgerät mit dem Server verbinden. Anzeigen und Ausblenden von Instrumenten und Anzeige von Daten testen. Anmeldung überprüfen.

\end{enumerate}

\end{document}