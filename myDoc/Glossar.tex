\documentclass[pflichtenheft.tex]{subfiles}

\begin{document}

\chapter{Glossar}

\begin{itemize}

\item
\textbf{Benutzer, Nutzer, User:} Jeder Nutzer der Software

\item
\textbf{Durchschnittsverbrauch:} Errechneter durchschnittlicher Verbrauch an Kraftstoff pro 100 Kilometer.

\item
\textbf{Dash: } Anzeigeelement, virtuelle Amatur

\item
\textbf{Dashboard: } Gesamtanzeige der Elemente; virtuelles Amaturenbrett

\item
\textbf{Endgerät:} Tablet, Smartphone, Laptop oder Desktop-PC.

\item
\textbf{Fahrer:} Der Benutzer, der das Fahrzeug zur Zeit fährt.

\item
\textbf{Fahrinformationen:} Die Gesamtheit aller für den aktuellen Fahrer während der Fahrt interessanten Sensordaten.

\item
\textbf{GUI, UI}: (Graphical) User Interface: Die grafische Benutzerschnittstelle der Software.

\item
\textbf{kritischer Sensorwert:} Sensorwert, der schwere Auswirkungen auf das zukünftige Verhalten oder die Funktion des Fahrzeugs haben kann.

\item
\textbf{POI:} Point of Interest; Ort in der Nähe, der (Aufgrund bestimmter Ereignisse) für den Benutzer interessant ist.

\item
\textbf{Rahmenwerk: } Softwaretechnisches Gerüst, in dem komponentenbasiert Funktionen hinzugefügt werden können. 

\item
\textbf{Remote Access:} Zugriff von einem Endgerät, das sich nicht in unmittelbarer Umgebung des Fahrzeugs befindet.

\item
\textbf{Server:} Dienstleister im Client-Server Modell.

\item
\textbf{System:} Die Gesamtheit der im Pflichtenheft beschriebenen Hardware und Software.

\end{itemize}

\end{document}