\documentclass[pflichtenheft.tex]{subfiles}

%% Dieses Dokument beschreibt Einleitung und Zielbestimmung

\begin{document}

\chapter{Einleitung}

In der modernen Informationsgesellschafft fallen immer mehr nutzbare Daten an. Seien es Daten von sozialen Netzwerken und Daten von Personen über Nutzung und Gewohnheiten allgemein, oder Daten technischer Art, wie Sensordaten in Anlagen oder Automobilen. Mehr und mehr Sensorik findet Einsatz. Oft ist es sehr interessant diese Daten auszuwerten und visuell zu repräsentieren. Tablets und Smartphones sind omnipräsent. Jeder hat dauerhaft Zugriff auf einen leistungsfähigen Rechner mit hochauflösendem Display. Diese Geräte sind also optimal zur Anzeige von Daten geeignet. Unabhängig von der Art der Daten müssen diese gesammelt, gegebenenfalls vorverarbeitet und bereitgestellt werden. Dazu bedarf es einer generischen Schnittstelle. In dem Rahmen von aRSDM sollen Sensordaten gesammelt, gespeichert und weiterverteilt werden. Die Quelle dieser Daten ist der mittlerweile in fast jedem PKW verfügbare OBD2 Anschluss. Es ist eine (abstrakte) Komponente zwischen Sensoren und Endgerät gefragt, die sich für beliebige, nicht unbedingt schon bekannte Anwendungsszenarien erweitern lässt. Ein Rahmenwerk, das Daten sammelt, speichert und verteilt.

\section{Aufgabenstellung}

Smartphones und Tablets des Users sind die sogenannten Endgeräte. Auf ihnen erfolgt eine Grafische Darstellung von Sensordaten. Die Quelle der Daten befindet sich in lokaler Distanz, die Daten gelangen über ein Netzwerk auf das Endgerät. aRSDM bietet eine API, die von Dritten für weitere Anwendungsfälle genutzt werden kann. Als Anwendung liefern wir ein Automotive-Modul, welches die Sensordaten, die über den CAN-Bus eines PKW verfügbar sind, vearbeitet und darstellt.


\chapter{Zielbestimmung}

Die Software aRSDM soll diejenigenl Sensordaten von einem Auto sammeln, die über den OBD2-Adapter verfügbar sind. aRSDM soll diese Daten speichern und Endgeräten zur Darstellung zur Verfügung stellen. Dabei soll eine API implementiert und genutzt werden, die auch auf weitere Anwendungen erweiterbar ist.

Die Darstellung soll in verschiedenen Internet-Browsern auf Endgeräten erfolgen. 

\section{Musskriterien}

%%Frage Lizensierung...
\renewcommand{\theenumi}{/MK\ifnum \value{enumi}<10 0\fi\arabic{enumi}0/}
\renewcommand{\labelenumi}{\theenumi}
\renewcommand{\theenumii}{\arabic{enumii}}
\renewcommand{\labelenumii}{/MK\ifnum \value{enumi}<10 0\fi\arabic{enumi}\arabic{enumii}/}

\begin{enumerate}

\item{\textbf{}} \\Der Nutzer kann Anzeigeelemente ein- und ausblenden.



\item{\textbf{Erweiterbarkeit}} \\Drittentwickler können Module für die Verarbeitung und Auswertung weiterer Sensordaren schreiben.

\item{\textbf{Verbindung}} \\Der Server soll in der Lage sein, entfernte Sensordaten zu Empfangen. Die Daten können über Bluetooth von einem OBD2 Bluetooth Adapter empfangen werden.



\item{\textbf{Fahrinformation}} Der Benutzer bekommt Fahrinformationen angezeigt.
\item{\textbf{Anzeigeelemente}} \\Der Nutzer kann Anzeigeelemente ein- und ausblenden.

\item{\textbf{RemoteAccess}} \\Falls Netzwerkverbindung besteht, können entfernte Nutzer die Sensordaten einsehen.

\item{\textbf{Protokollierung}} \\Der Nutzer soll jederzeit in der Lage sein, Daten von vergangenen Zeitpunkten abzurufen.

\item{\textbf{aggregierte Funktionen}} Die Software soll aus verschiedenen Sensordaten aggregierte Funktionen wie Durchschnittsverbrauch und Restkilometer berechnen.

\item{\textbf{Einparkhile}} Dem Fahrer steht eine Einparkhilfe mit Augmented Reality-Features zur Verfügung.

\item{\textbf{POI}} Bei kritischen Sensorwerten kann der Nutzer die nächsten Tankstellen / Werstätten einbleden.
\end{enumerate}

\renewcommand{\theenumi}{/WK\ifnum \value{enumi}<10 0\fi\arabic{enumi}0/}
\renewcommand{\labelenumi}{\theenumi}
\renewcommand{\theenumii}{\arabic{enumii}}
\renewcommand{\labelenumii}{/WK\ifnum \value{enumi}<10 0\fi\arabic{enumi}\arabic{enumii}/}
\section{Wunschkriterien}

\begin{enumerate}
\item{\textbf{Smartphone Utilization}} Die Software verwendet auch Sensordaten aus dem Endgerät.

\item{\textbf{Personalisierung}} Aggregierte Funktionen sind personalisiert verfügbar.

\item{\textbf{Fahrer}} Ein Nutzer, der mit dem Server verbunden ist, kann sich mit der Software auf seinem Endgerät als Fahrer identifizieren.
\end{enumerate}

\renewcommand{\theenumi}{/AK\ifnum \value{enumi}<10 0\fi\arabic{enumi}0/}
\renewcommand{\labelenumi}{\theenumi}
\renewcommand{\theenumii}{\arabic{enumii}}
\renewcommand{\labelenumii}{/AK\ifnum \value{enumi}<10 0\fi\arabic{enumi}\arabic{enumii}/}

\section{Abgrenzungskriterien}

\begin{enumerate}
\item{\textbf{Erweiterung}} Für andere Sensorquellen werden keine Treiber bereitgestellt.

\item{\textbf{Unidirektional}} Die Software liest und interpretiert die Daten nur und sendet keine Daten an Quellen zurück.

\item{\textbf{Bildauswertung}} Kamerabilder werden von der Software nicht interpretiert.

\item{\textbf{Android}} Die Visualisierung der Daten erfolgt nur im Browser, wir stellen keine gesonderte App zur Verfügung.

\item{\textbf{Flotte}} Der Zugriff auf verschiedene Fahrzeuge wird nicht verwaltet.

\item{\textbf{Navigation}} Die Anzeige der aktuellen Position und der umgebenden POI erfolgt über Drittsoftware und kann nicht als Navigationssystem benutzt werden.

\item{\textbf{Diagnose}} Die Datenspeicherung ersetzt nicht den Fehlerspeicher des Steuergeräts im Auto.

\item{\textbf{RemoteAcces}} Für entferntes Monitoring wird keine maximale Latenz garantiert 

\end{enumerate}



\end{document}