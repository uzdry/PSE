\documentclass[pflichtenheft.tex]{subfiles}

%% Dieses Dokument beschreibt Einleitung und Zielbestimmung

\begin{document}

\chapter{Einleitung}

In der modernen Informationsgesellschafft fallen immer mehr nutzbare Daten an. Seien es Daten von sozialen Netzwerken und Daten von Personen über Nutzung und Gewohnheiten allgemein, oder Daten technischer Art, wie Sensordaten in Anlagen oder Automobilen. Mehr und mehr Sensorik findet Einsatz. Oft ist es sehr interessant diese Daten auszuwerten und visuell zu repräsentieren. Tablets und Smartphones sind omnipräsent. Jeder hat dauerhaft Zugriff auf einen leistungsfähigen Rechner mit hochauflösendem Display. Diese Geräte eignen sich also prima zur Anzeige von Daten. Unabhängig von der Art der Daten, müssen sie gesammelt, evtl. vorverarbeitet und bereitgestellt werden. Dazu bedarf es einer generischen Schnittstelle. In dem Rahmen von aRSDM sollen Sensordaten gesammelt und verteilt werden. Genauer gesagt, die über den OBD2 Anschluss verfügbare Daten in einem PKW. Es ist eine (abstrakte) Komponente zwischen Sensoren und Endgerät gefragt, die sich für beliebige, nicht unbedingt schon bekannte Anwendungsszenarien erweitern lässt. Ein Rahmenwerk, das Daten sammelt, speichert und verteilt.

\section{Aufgabenstellung}

Smartphones und Tablets des Users sind die sogenannten Endgeräte. Auf ihnen erfolgt eine Grafische Dartstellung von Sensordaten. Die Quelle der Daten ist entfernt und die Daten werden gelangen über Netzwerk auf das Smartphone. RSDM bietet eine API, die von Dritten für weitere Szenarien genutzt werden kann. Als Anwendung liefern wir ein automotive Modul, welches Sensordaten, die über den CAN Bus eine PKW verfügbar sind, vearbeitet.


\chapter{Zielbestimmung}

Die Software aRSDM soll Sensordaten von einem Auto sammeln, die über den CAN Bus auszulesen sind speichern und Endgeräten zur Darstellung zur Verfügung stellen. Dabei soll eine API implementiert und genutzt werden, die auch auf weitere Anwendungen erweiterbar ist.

Es soll die Darstellung im Browser Google Chrome\textsuperscript{TM} auf Endgeräten erfolgen. 

\section{Musskriterien}

%%Frage Lizensierung...

\myparagraph{Allgemeine Funktionen}

\subsection{\mknr Protokollierung} Der Nutzer soll Daten aus der Vergangenheit zu jeder Zeit abrufen können.


\subsection{\mknr Erweiterbarkeit} Drittentwickler können Module für die Verarbeitung und Auswertung weiterer Sensordaren schreiben.


\subsection{\mknr RemoteAcces} Falls Netzwerkverbindung besteht, können entfernte Nutzer die Sensordaten einsehen.

\subsection{\mknr Anzeigeelemente} Der Nutzer kann Anzeigeelemente ein- und ausblenden.

\subsection{\mknr Statistiken} Der Benutzer kann sich Statistiken der gespeicherten Sensorwerte anzeigen lassen.

\myparagraph{Automotive Kontext}

\subsection{\mknr Fahrinformation} Der Benutzer bekommt Fahrinformation (Glossar) angezeigt.

\subsection{\mknr Durschnitssverbrauch} Die Software zeigt den Durschnittsverbrauch des Fahrzeugs an. 

\subsection{\mknr Restkilometer} Die Software zeigt die verfügbaren Restkilometer bis zur nächsten Tankfüllung an. Diese werden aufgrungd des Fahrverhaltens berechnet.

\subsection{\mknr Einparkhile} Dem Fahrer steht eine Einparkhilfe mit Augmented Reality Features zur Verfügung.

\subsection{\mknr POI} Bei bestimmten Sensorwerten (Tank/Motor) kann der sich Nutzer die nächsten Tankstellen / Werstätten einbleden.


\section{Wunschkriterien}


\subsection{\mknr Smartphone Utilization} Die Software verwendet auch Sensordaten aus dem Endgerät.

\subsection{\mknr Personalisierung} Aggregierte Funktionen sind personalisiert verfügbar.

\section{Abgrenzungskriterien}

\subsection{\mknr Erweiterung} Für andere Sensorquellen werden keine Treiber bereitgestellt.

\subsection{\mknr Unidirektional} Die Software liest und interpretiert Daten, sendet keine Daten an Quellen zurück.

\subsection{\mknr Bildauswertung} Karabilder werden von der Software nicht interpretiert.

\subsection{\mknr Android} Die Visualisierung der Daten erfolgt nur im Browser, wir stellen keine gesonderte App zur Verfügung.


\end{document}