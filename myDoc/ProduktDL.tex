\documentclass[pflichtenheft.tex]{subfiles}

\begin{document}

\chapter{Produktdaten}

\renewcommand{\theenumi}{/PD\ifnum \value{enumi}<10 0\fi\arabic{enumi}0/}
\renewcommand{\labelenumi}{\theenumi}
\renewcommand{\theenumii}{\arabic{enumii}}
\renewcommand{\labelenumii}{PD\ifnum \value{enumi}<10 0\fi\arabic{enumi}\arabic{enumii}/}

% Start of new text 1/2.
\section{Physische Sensordaten}
Es sind zeitgestempelte und fahrergestempelte (~\ref{driver1}) Messwerte der folgenden Sensoren zu speichern:
\begin{enumerate}
\item
Temperatur der Motorkühlflüssigkeit
\item
Kraftstoffdruck
\item
Motordrehzahl
\item
Geschwindigkeit des Fahrzeugs
\item
Lenkraddrehung
\item
Ansauglufttemperatur 
\item
Laufzeit des Motors seit letztem Start
\item
Tankfüllstand
\item
Status der Abgasrückführung
\item
Status des Kontrollsystems für Einspritzdruck
\item
Status des Kontrollsystems für Kraftstoffdruck
\item
Gasdruck des Verdampfers
\item
Katalysatortemperatur
\item
Position der Drosselklappe
\item
Position des Gaspedals
\item
Außentemperatur
\item
Motordrehmoment
\item
Abgastemperatur
\item
Abgasdruck
\item
Distanzwerte der Ultraschallsensoren
\item Durchschnittsverbrauch ( temporär )
\item Zurückgelegte Strecke ( temporär )
\item Durschnittsgeschwindigkeit ( temporär )
\setcounter{enumTemp}{\value{enumi}}
\end{enumerate}

\section{Anwendungsinformtionen}
Es sind die folgenden Daten der Anwendungsinformationen zu speichern:

\begin{enumerate}
\setcounter{enumi}{\value{enumTemp}}
\item Endgerät-Signatur
\item Anzeigeeinstellungen
\item Präferierter Kraftstoff (Super E5, Diesel, etc.)
\item Anzahl der Einzelfahrten
\item Bluetooth-Verbindungsdaten mit dem OBD2-Adapter ( temporär )
\item Netzwerk-Verbindungsdaten aller verbundenen Geräte ( temporär ) 
\item Aktuell anzuzeigende Sensordaten auf den Endgeräten ( temporär )
\item Aktuell als Fahrer eingetragener Benutzer ( temporär )
\item Aktuelle Fahrtnummer ( temporär )
\setcounter{enumTemp}{\value{enumi}}
\end{enumerate}


\setcounter{enumTemp}{\value{enumi}}
\end{enumerate}

\chapter{Produktleistungen}

\renewcommand{\theenumi}{/PL\ifnum \value{enumi}<10 0\fi\arabic{enumi}0/}
\renewcommand{\labelenumi}{\theenumi}
\renewcommand{\theenumii}{\arabic{enumii}}
\renewcommand{\labelenumii}{PL\ifnum \value{enumi}<10 0\fi\arabic{enumi}\arabic{enumii}/}

\begin{enumerate}

	\item{\textbf{Latenz von~\ref{pastData}}} \\
	Die Funktion~\ref{pastData} soll 300 ms dauern. Wenn die Daten nach dieser Zeit noch nicht angezeigt wurden soll ein Wartesymbol angezeigt werden. 

	\item{\textbf{Auswirkung von~\ref{deleteData}}} \\
	Die Funktion~\ref{deleteData} darf keine Auswirkungen auf die sonstige Performanz des Systems haben.

	\item{\textbf{Latenz von~\ref{liveData}}} \\
	Die Funktion~\ref{liveData} darf höchstens 100 ms benötigen.

	\item{\textbf{Bildrate Rückfahrkamera}} \\
	Das Kamerabild sowie die dargestellte Fahrbahn sollen im zeitlichen Abstand von 100 ms aktualisiert werden.

	\item{\textbf{Aktualisierungsrate Ultraschallsensoren}} \\
	Die Aktualisierung von~\ref{ultrasound} soll im zeitlichen Abstand von 100 ms stattfinden.

	\item{\textbf{Multiple Access}} \\
	Die Verbindung von bis zu vier Personen gleichzeitig ist möglich.


\end{enumerate}

\end{document}