\documentclass[pflichtenheft.tex]{subfiles}

\begin{document}

\chapter{Produktdaten}

\renewcommand{\theenumi}{/PD\ifnum \value{enumi}<10 0\fi\arabic{enumi}0/}
\renewcommand{\labelenumi}{\theenumi}
\renewcommand{\theenumii}{\arabic{enumii}}
\renewcommand{\labelenumii}{PD\ifnum \value{enumi}<10 0\fi\arabic{enumi}\arabic{enumii}/}

% Start of new text 1/2.
\section{\mkd physische Sensordaten}
Es sind zeitgestempelte und fahrergestempelte Messwerte der folgenden Sensoren zu speichern:
\begin{enumerate}
\item
Temperatur der Motorkühlflüssigkeit
\item
Kraftstoffdruck
\item
Motordrehzahl
\item
Geschwindigkeit des Fahrzeugs
\item
Lenkraddrehung
\item
Ansauglufttemperatur 
\item
Laufzeit des Motors seit letztem Start
\item
Tankfüllstand
\item
Status der Abgasrückführung
\item
Status des Kontrollsystems für Einspritzdruck
\item
Status des Kontrollsystems für Kraftstoffdruck
\item
Gasdruck des Verdampfers
\item
Katalysatortemperatur
\item
Position der Drosselklappe
\item
Position des Gaspedals
\item
Außentemperatur
\item
Motordrehmoment
\item
Abgastemperatur
\item
Abgasdruck
\item
Distanzwerte der Ultraschallsensoren
\setcounter{enumTemp}{\value{enumi}}
\end{enumerate}

\section{\mkd virtuelle Sensordaten}
Es sind Werte der folgenden virtuellen Sensoren zu speichern:

\begin{enumerate}
\setcounter{enumi}{\value{enumTemp}}
\item Anzahl der Einzelfahrten
\item Aktueller Treibstoffverbrauch
\setcounter{enumTemp}{\value{enumi}}
\end{enumerate}

\section{\mkd Personenbezogene Daten}
Es sind Profilinformationen der Fahrer zu speichern:

\begin{enumerate}
\setcounter{enumi}{\value{enumTemp}}
\item Endgerät-Signatur
\item Anzeigeeinstellungen
\item Genutzter Kraftstoff
\setcounter{enumTemp}{\value{enumi}}
\end{enumerate}

\section{\mkd Laufzeitdaten}
Zur Laufzeit sind weiterhin folgende Daten temporär zu speichern:
\begin{enumerate}
\setcounter{enumi}{\value{enumTemp}}
\item Bluetooth-Verbindungsdaten mit dem OBD2-Adapter
\item Netzwerk-Verbindungsdaten aller verbundenen Geräte
\item Aktuell anzuzeigende Sensordaten auf den Endgeräten
\item Aktuell als Fahrer eingetragener Benutzer
\item Aktuelle Fahrtnummer
\item Durchschnittsverbrauch
\item Zurückgelegte Strecke
\item Durschnittsdauer/ Strecke
\setcounter{enumTemp}{\value{enumi}}
\end{enumerate}

\chapter{Produktleistungen}

\renewcommand{\theenumi}{/PL\ifnum \value{enumi}<10 0\fi\arabic{enumi}0/}
\renewcommand{\labelenumi}{\theenumi}
\renewcommand{\theenumii}{\arabic{enumii}}
\renewcommand{\labelenumii}{PL\ifnum \value{enumi}<10 0\fi\arabic{enumi}\arabic{enumii}/}

\begin{enumerate}

\item{\textbf{Latenz von ~\ref{pastData}}} \\
Die Funktion ~\ref{pastData} soll 300 ms dauern. Wenn die Daten nach dieser Zeit noch nicht angezeigt wurden soll ein Wartesymbol angezeigt werden. 

\item{\textbf{Auswirkung von ~\ref{deleteData}}} \\
Die Funktion ~\ref{deleteData} darf keine Auswirkungen auf die sonstige Performanz des Systems haben.

\item{\textbf{Latenz von ~\ref{liveData}}} \\
Die Funktion ~\ref{liveData} darf höchstens 100 ms benötigen.

\item{\textbf{Bildrate Rückfahrkamera}} \\
Das Kamerabild sowie die dargestellte Fahrbahn sollen im Abstand von 100 ms aktualisiert werden.

\item{\textbf{Aktualisierungsrate Ultraschallsensoren}} \\
Die Aktualisierung von ~\ref{ultrasound} soll im Abstand von 100 ms stattfinden.

\item{\textbf{Multiple Access}} \\
Die Verbindung von bis zu vier Personen gleichzeitig ist möglich.


\end{enumerate}

\end{document}