\documentclass[pflichtenheft.tex]{subfiles}

\begin{document}

\chapter{Produktdaten}

% Start of new text 1/2.
\section{\mkd physische Sensordaten}
Es sind zeitgestempelte und fahrergestempelte Messwerte der folgenden Sensoren zu speichern:
\begin{itemize}
\item
Temperatur der Motorkühlflüssigkeit
\item
Kraftstoffdruck
\item
Motordrehzahl
\item
Geschwindigkeit des Fahrzeugs
\item
Lenkraddrehung
\item
Ansauglufttemperatur 
\item
Laufzeit des Motors seit letztem Start
\item
Tankfüllstand
\item
Status von
\begin{itemize}
\item Abgasrückführung
\item Kontrollsystem für Einspritzdruck
\item Kontrollsystem für Kraftstoffdruck
\end{itemize}
\item
Gasdruck des Verdampfers
\item
Katalysatortemperatur
\item
Position der Drosselklappe
\item
Position des Gaspedals
\item
Außentemperatur
\item
Motordrehmoment
\item
Abgastemperatur
\item
Abgasdruck
\item
Distanzwerte der Ultraschallsensoren

\end{itemize}

\section{\mkd aggregierte Sensordaten}
Es sind Werte der folgenden virtuellen Sensoren zu speichern:
\begin{itemize}
\item Anzahl der Einzelfahrten
\item Aktueller Treibstoffverbrauch


\end{itemize}

\section{\mkd Personenbezogene Daten}
Es sind Profilinformationen der Fahrer zu speichern:
\begin{itemize}
\item
Endgerät-Signatur
\item
Anzeigeeinstellungen

\end{itemize}

\section{\mkd Laufzeitdaten}
Zur Laufzeit sind weiterhin folgende Daten temporär zu speichern:
\begin{itemize}
\item Bluetooth-Verbindungsdaten mit dem OBD2-Adapter
\item Netzwerk-Verbindungsdaten aller verbundenen Geräte
\item Aktuell anzuzeigende Sensordaten auf den Endgeräten
\item Aktuell als Fahrer eingetragener Benutzer
\item Aktuelle Fahrtnummer
\item Durchschnittsverbrauch
\item Zurückgelegte Strecke
\item Durschnittsdauer/strecke

\end{itemize}


% End of new text 1/2.

\chapter{Produktleistungen}

%Start of new text 2/2.
\subsection{\mkl Einparkhilfe-Verzögerung}
Die Verzögerung des Informationsflusses darf maximal 100 ms betragen. Dabei sind die Netzwerkverzögerungen (Bluetooth, LAN ,Internet...) zwischen den verschiedenen Geräten nicht berücksichtigt. 

\subsection{\mkl Kamerabildaktualisierung}
Das Kamerabild sowie die dargestellte Fahrbahn sollen mindestens 10-mal pro Sekunde aktualisiert werden.
%End of new text 2/2.

\subsection{\mkl Multiple Acces} Die Verbindung von bis zu vier Personen gleichzeitig ist möglich.

\end{document}