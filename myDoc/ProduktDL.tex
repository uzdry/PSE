\documentclass[pflichtenheft.tex]{subfiles}

\begin{document}

\chapter{Produktdaten}

% Start of new text 1/2.
\section{\mkd physische Sensordaten}
Es sind zeitgestempelte und fahrergestempelte Messwerte der folgenden physischen Sensoren zu speichern:
\begin{itemize}
\item
Ultraschallsensoren
\item
Lenkraddrehung
\item
...

\end{itemize}

\section{\mkd aggregierte Sensordaten}
Es sind Werte der folgenden virtuellen Sensoren zu speichern:
\begin{itemize}
\item
Fahrtdauer
\item
Anzahl der Einzelfahrten
\item
...

\end{itemize}

\section{\mkd Personenbezogene Daten}
Es sind Profilinformationen der Fahrer zu speichern:
\begin{itemize}
\item
Smartphone-Signatur
\item
Anzeigeeinstellungen
\item
...
\end{itemize}

\section{\mkd Laufzeitdaten}
Zur Laufzeit sind weiterhin folgende Daten zu speichern:
\begin{itemize}
\item
Bluetooth-Verbindungsdaten mit dem OBD2-Adapter
\item
Netzwerk-Verbindungsdaten aller verbundenen Geräte
\item
Aktuell anzuzeigende Sensordaten auf den Endgeräten


\end{itemize}


% End of new text 1/2.

\chapter{Produktleistungen}

%Start of new text 2/2.
\subsection{\mkl Einparkhilfe-Verzögerung}
Die Verzögerung des Informationsflusses darf maximal 100 ms betragen. Dabei sind die Netzwerkverzögerungen (Bluetooth, LAN ,Internet...) zwischen den Verschiedenen Geräten nicht berücksichtigt. 

\subsection{\mkl Kamerabildaktualisierung}
Das Kamerabild sowie die dargestellte Fahrbahn sollen mindestens 10-mal pro Sekunde aktualisiert werden.
%End of new text 2/2.

\subsection{\mkl Multiple Acces} Die Verbindung von bis zu vier Personen gleichzeitig ist möglich.

\end{document}