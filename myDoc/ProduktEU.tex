\documentclass[pflichtenheft.tex]{subfiles}

\begin{document}

%-------------- PRODUKTEINSATZ ---------------------
\chapter{Produkteinsatz}
Das Produkt dient zur Speicherung und Distribution von Daten mehrerer verteilter Sensoren eines Autos. Dies beinhaltet z.B. Informationen wie die Geschwindigkeit, Motortemperatur, Raddrehzahl oder ähnliches. Damit bietet es die Möglichkeit diese Daten z.B. auf einem Handydisplay oder einem Dashboard anzuzeigen


\section{Anwendungsbereiche}
\begin{itemize}
\item
Dieses Produkt dient zum Speichern und Anzeigen von Sensordaten, die im Auto verfügbar sind.
\end{itemize}


\section{Zielgruppen}
\begin{itemize}
\item
Private Autofahrer mit oder ohne tiefere technische Kenntnisse. Firmen mit Automobilflotten. 
\end{itemize}


\section{Betriebsbedingungen}
\begin{itemize}
\item
Auto muss über OBD2 Anschluss verfügen.
\item
Es wird über den Autostrom laufen.
\item
Für Remote Zugriff muss Netzwerkverbindung vorhanden sein.
\end{itemize}


%--------------- Produktumgebung -------------
\chapter{Produktumgebung}
Das Rahmenwerk und die Module werden in TypeScript programmiert. Der Server, in node.js. Als Datenbanksystem wird LevelUp verwendet.

\begin{itemize}

\item
Eine Client-Server Architektur mit dem Thin Client Konzept
\item
Auf dem Server läuft der Teil der Software, der die Sensordaten empfängt, einpflegt und über WebRTC an die Endgeräte versendet.
\item
Auf der Userseite laufen einige verschiedene Applikationen neben Unserer. Diese sollen sich jedoch gegenseitig nicht beeinflussen.
\end{itemize}

%%%%Fragen
\section*{Testumgebung}
\begin{itemize}
\item
Testfahrzeug: Opel Astra H 1.6 Bj 2005 %%Genauer
\item
Modellauto Nicolas\\
Ein Modellauto der Firma ITK-Engineering AG, angetrieben über einen Elektromotor. Jegliche Daten, unter anderem von Ultraschallsensoren, liegen über CAN an
\end{itemize}


\section{Software}
\begin{itemize}
\item
Serverseite\\
Ein Webserver mit einer Datenbank\\
Aufgebaut auf einer Linux-Distribution
\item
Clientseite\\
Web Browser:
\begin{itemize}
\item
Google Chrome Version 46
\item
Safari Version 9
\item
Firefox Version 42
\item
Google Chrome Android Version 14.10
\end{itemize}
Getestet mit:
\begin{itemize}
\item
Mac OS X
\item
Fedora 22
\item
Windows 7
\end{itemize}
\end{itemize}


\section{Hardware}
\begin{itemize}
\item
Serverseite\\
Kleincomputer mit:
\begin{itemize}
\item
Broadcom BCM2836 Arm7 Quad Core Processor (900MHz)
\item
1GB RAM
\item
4 x USB 2 ports
\end{itemize}
Can-Modul
\item
Clientseite\\
Standardrechner (min. 1 GHz und 2 GB Ram)\\
Android Phone (min. 1 GHz und 1 GB Ram)
\end{itemize}
\end{document}



