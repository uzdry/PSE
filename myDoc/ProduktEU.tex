\documentclass[pflichtenheft.tex]{subfiles}

\begin{document}

%-------------- PRODUKTEINSATZ ---------------------
\chapter{Produkteinsatz}
Das Produkt dient zur Speicherung und Distribution von Daten mehrerer verteilter Sensoren eines Autos. Dies beinhaltet z.B. Informationen wie die Geschwindigkeit, Motortemperatur, Raddrehzahl oder ähnliches. Damit bietet es einerseits einem Endnutzer den Zugriff auf aktuelle und gespeicherte Zustände des Automobils, außerdem würde es z.B. ermöglichen, dass bei einer Reparatur diese Daten zur Fehleranalyse genutzt werden können.


\section{Anwendungsbereiche}
\begin{itemize}
\item
Privater oder geschäftlicher Anwendungsbereich
\item
Automobil Anwendungsbereich
\end{itemize}


\section{Zielgruppen}
\begin{itemize}
\item
Private sowie geschäftliche Autobesitzer mit oder ohne tiefere technische Kenntnisse
\item
Mitarbeiter in einer Autowerkstatt
\item
Automobil-Entwickler
\end{itemize}


\section{Betriebsbedingungen}
\begin{itemize}
\item
Einsatz im üblichen Aufgabengebiet eines Autos
\item
Einsatz beim Entwickeln oder Testen neuer Automobiltechnologien
\end{itemize}


%--------------- Produktumgebung -------------
\chapter{Produktumgebung}
\begin{itemize}
\item
Eine Client-Server Architektur mit dem ``Thin Client`` Konzept
\item
Auf dem Server läuft hauptsächlich unsere Anwendung und das Einlesen der Sensordaten.
\item
Auf der Userseite laufen einige verschiedene Applikationen neben Unserer. Diese sollen sich jedoch gegenseitig nicht beeinflussen.
\end{itemize}


\section{Software}
\begin{itemize}
\item
Serverseite\\
Datenbank auf Unix, läuft lokal im Auto\\
Webserver: tomcat oder apache
\item
Clientseite\\
Betriebssystem: Windows 7, 8, 10, Verschiedene Linux Distributionen, Mac OS X, Android 4 und später\\
Web Browser, Referenz Google Chrome Version 46
\end{itemize}


\section{Hardware}
\begin{itemize}
\item
Serverseite\\
Raspberry Pi
\item
Clientseite\\
Standardrechner (min. 1 GHz und 2 GB Ram)\\
Android Phone (min. 1 GHz und 1 GB Ram)
\end{itemize}
\end{document}



