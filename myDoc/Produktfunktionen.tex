\documentclass[pflichtenheft.tex]{subfiles}

\begin{document}

\chapter{Produktfunktionen}

\subsection{Allgemeine Funktionen}

\begin{enumerate}
	\item{Protokoll} 

	\item{Datenhaltung} Die definierte maximale Kapazität der Datenbank ist erreicht. Die ältesten Daten werden nach erreichen der maximalen Kapazität der Datenbank gelöscht.
	Nachbedingung: Datenbank hat weniger Einträge als ihre maximale Kapazität. Die ältesten Einträge wurden gelöscht.
	\begin{enumerate}
		\item cc
	\end{enumerate} 
	\begin{enumerate}
		\item Der Nutzer stellt maximalen Wert einstellen.
	\end{enumerate}

\end{enumerate}

\subsection{\mkfa Protokoll} Der Nutzer soll Daten aus der Vergangenheit zu jeder Zeit abrufen können.

\subsection{\mkfa Statistiken} Der Benutzer kann sich Statistiken der gespeicherten Sensorwerte anzeigen lassen.

\subsection{\mkfa Erweiterbarkeit} Drittentwickler können Module für die Verarbeitung und Auswertung weiterer Sensordaren schreiben.

\subsection{\mkfa RemoteAcces} Falls Netzwerkverbindung besteht, können entfernte Nutzer die Sensordaten einsehen.

\subsection{\mkfa Anzeigeelemente} Der Nutzer kann Anzeigeelemente für jeden der Sensorwerte ein- und ausblenden.

\subsection{Automotive Kontext}

\subsection{\mkfa Fahrinformation} Der Benutzer bekommt Fahrinformation (Glossar) angezeigt.

\subsection{\mkfa Boardcomputer} Die Software zeigt berechnet anhand der Sensorwerte aggregierte Funktionen. Wie Durchschnittsverbrauch.

\subsubsection{\mkfa Restkilometer} Die Software zeigt die verfügbaren Restkilometer bis zur nächsten Tankfüllung an. Diese werden aufgrungd des Fahrverhaltens berechnet.

\subsection{\mkfa Einparkhile} Dem Fahrer steht eine Einparkhilfe mit Augmented Reality Features zur Verfügung.

\subsection{\mkfa POI} Bei bestimmten Sensorwerten (Tank/Motor) kann der sich Nutzer die nächsten Tankstellen oder Werstätten einbleden.

\subsection{\mkfa Dateninput}
Wenn Daten über Bluetooth ankommen pflegt die Software diese Daten in eine Datenbank ein. 
\subsection{\mkfa Vergangene-Daten}
Wenn der Nutzer es wünscht zeigt die Software ihm die gespeicherten Daten an.
\subsection{\mkfa Zustandswarnungen}
Das System überwacht und reagiert auf kritische Sensorwerte und Warnungen.

\subsection{\mkfa Karte}
Der Benutzer kann sich vom System einen aktuellen Kartenausschnitt der Umgebung anzeigen lassen.
\subsubsection{\mkfa Werkstatt/Tankstellen}
Die Software kann auf dieser Karte auch Werkstätte, Tankstellen und Restaurants anzeigen.
\subsubsection{\mkfa Niedriger Tankfüllstand}
Bei niedrigem Tankfüllstand werden beim öffnen dieser Karte automatisch die Tankstellen in der Nähe angezeigt.
\subsubsection{\mkfa Weitere Probleme} % oder zu heißem Öl 
Bei weiteren Problemen wie einer Kontrollleuchte werden beim öffnen dieser Karte die Werkstätten in der Nähe angezeigt.
\subsection{\mkfa Einparkhilfe}
Beim rückwärts Einparken aktiviert die Software eine Rückfahrkamera und indiziert die aktuelle Fahrbahn.
\subsubsection{\mkfa Ultraschallsensoren}
Außerdem werden die Ultraschallsensordaten angezeigt.

\section{Wunsch}

\subsection{\mkfaw Smartphone Utilization} Die Software verwendet auch Sensordaten aus dem Endgerät.
\subsection{\mkfaw Personalisierung} Aggregierte Funktionen werden aufgrund der vorhandenen Daten des aktuell als Fahrer angemeldeten Nutzers bereitgestellt.

\subsubsection{\mkfaw Fahrer} Ein Nutzer, der mit dem Server verbunden ist, kann sich mit der Software auf seinem Endgerät als Fahrer identifizieren (/FA0070/, /FA1020W/).
\end{document}