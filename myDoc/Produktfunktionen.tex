\documentclass[pflichtenheft.tex]{subfiles}

\begin{document}

\chapter{Produktfunktionen}

\renewcommand{\theenumi}{/FA\ifnum \value{enumi}<10 0\fi\arabic{enumi}0/}
\renewcommand{\labelenumi}{\theenumi}
\renewcommand{\theenumii}{\arabic{enumii}}
\renewcommand{\labelenumii}{/FA\ifnum \value{enumi}<10 0\fi\arabic{enumi}\arabic{enumii}/}

\section{Endgerät mit Server verbinden}

	\begin{enumerate}
		\item{\textbf{Erstmals anmelden}} \label{firstcon}
		\\ Ziel: Der Benutzer soll sich im System anmelden können, wobei sich nur maximal eine Person als Fahrer anmelden kann. \\
		Vorbedingung: Das Endgerät war noch nicht mit dem Server verbunden.\\
		Nachbedingung: Es wird ein Account für den Nutzer dieses Endgeräts angelegt.\\
		Auslösendes Ereignis: Es wird eine Netzwerkverbindung zwischen Server und Endgerät aufgebaut.

		\item{\textbf{Erneut anmelden}} \label{connection}
		\\Vorbedingung: Das Endgerät war bereits mit dem Server verbunden und ist zum aktuellen Zeitpunkt getrennt. \\
		Nachbedingung: Das Gerät ist angemeldet und der entsprechende Account ist ausgewählt.\\
		Auslösendes Ereignis: Es wird eine Netzwerkverbindung zwischen Server und Endgerät aufgebaut.

		\item{\textbf{Nutzer als Fahrer identifizieren}} \label{driver1}
		\\ Ziel: Genau ein Benutzer wird vom System als Fahrer erkannt.\\
		Vorbedingung: Kein Endgerät ist mit dem Server verbunden.\\
		Nachbedingung: Der Benutzer dieses Geräts wird als Fahrer identifiziert. \\
		Auslösendes Ereignis: Der Benutzer meldet sich als Fahrer an.

		\item{\textbf{Fahrer unter Nutzern identifizieren}} \label{driver2}
		\\ Ziel: Höchstens ein Benutzer, der sich mit einem anderen Endgerät mit dem Server verbindet, wird vom System als Fahrer erkannt.  \\
		Vorbedingung: Es ist mindestens ein Endgerät mit dem Server verbunden.\\
		Nachbedingung: Der Benutzer eines anderen Endgeräts wird als Fahrer identifiziert. \\
		Auslösendes Ereignis: Der Benutzer verbindet ein weiteres Endgerät mit dem Server und meldet sich als Fahrer an.

		\setcounter{enumTemp}{\value{enumi}}
		

	\end{enumerate}
	%GUI-Elemente einblenden:
	
	\section{GUI-Instrumente auswählen}
	Der Nutzer soll GUI-Instrumente anzeigen und ausblenden können. Zu den GUI-Instrumenten gehören sowohl Dashes mit Echtzeitinformationen als auch Statistiken. GUI-Elemente können nur zu im Fahrzeug verfügbaren Daten eingeblendet werden. Einige GUI-Elemente werden immer angezeigt.\\
	Vorbedingung: Der Server läuft und es sind Endgeräte mit ihm verbunden.
	
	\begin{enumerate}
	\setcounter{enumi}{\value{enumTemp}}
		\item{\textbf{Instrumente anzeigen}} \\Vorbedingung: - \\ Nachbedingung: Das Instrument wird auf dem Bildschirm angezeigt und zeigt die aktuellen Daten an. Falls keine aktuellen Daten vorhanden sind, steht es in neutraler Position. \\ Auslösendes Ereignis: Der Nutzer wählt in den Einstellungen auf dem Endgerät ein Instrument aus.

		\item{\textbf{Instrumente verfügbar machen}} \\
		Ziel: Es werden nur Instrumente verfügbar gemacht, zu denen Daten im Auto verfügbar sind. \\
		Vorbedingung: Die Information, welche Daten im Fahrzeug verfügbar sind, ist vorhanden. \\
		Nachbedingung: Nur diejenigen Instrumente, deren Daten verfügbar sind, werden angezeigt.

		\item{\textbf{Dashboard nicht überladen}} \\Ziel: Das Dashboard soll nicht überladen wirken.\\
		Auslösendes Ereignis: Der Nutzer wählt in den Einstellungen ein zusätzliches Instrument aus, welches nicht mehr passen würde.\\ Nachbedingung: Das Dashboard wird nach rechts erweitert. \\ Vorbedingung: Es werden Dashes auf dem Dashboard angezeigt.

		\item{\textbf{Instrumente ausblenden}} \\
		Ziel: Der Nutzer kann bestimmte Anzeigeelemente ausblenden.\\
		Vorbedingung: Es werden Dashes auf dem Dashboard angezeigt.\\
		Nachbedingung: Das Instrument wird nicht mehr angezeigt.\\
		Auslösendes Ereignis: Der Benutzer wählt im Einstellungspanel ein Dash ab.
		\setcounter{enumTemp}{\value{enumi}}
	\end{enumerate} 
	%Datenhaltung:
	
	\section{Datenhaltung}

	\begin{enumerate}
		\setcounter{enumi}{\value{enumTemp}}
		\item{\textbf{Protokoll führen}} \\ Ziel: Es werden eingehende Daten gespeichert.\\
		Vorbedingung: Bluetooth-Verbindung mit OBD2-Adapter ist hergestellt. \\
		Nachbedingung: Die Daten, die von der Bluetooth-Schnittstelle kommen, befinden sich in der Datenbank. \\ 
		Auslösendes Ereignis: Von der Bluetooth-Schnittstelle treffen Daten ein.
		
		\item{\textbf{Vergangene Daten aufrufen}} \label{pastData} \\ Ziel: Es werden Daten von vergangenen Zeitpunkten dargestellt. \\ Vorbedingung: Es wurde bereits eine Fahrt getätigt.\\
		Nachbedingung: Die Daten werden auf dem Endgerät angezeigt. \\
		Auslösendes Ereignis: Der Nutzer fordert Daten aus der Vergangenheit an.\\
		\item{\textbf{Maximale Kapazität einstellen}}\\
		Ziel: Es soll eine neue maximale Kapazität der Datenbank gesetzt werden. \\ Vorbedingung: keine weiteren Vorbedingungen.\\
		Nachbedingung: Die neue gespeicherte maximale Kapazität der Datenbank ist der vom Nutzer eingegebene Wert.\\
		Auslösendes Ereignis: Der Benutzer versucht, eine neue maximale Kapazität der Datenbank zu setzen. 
		
		\item{\textbf{Daten automatisch löschen}} \label{deleteData} \\ Ziel: Um Überfüllung der Datenbank zu vermeiden, werden Daten gelöscht. \\ Vorbedingung: keine weiteren Vorbedingungen.\\ Nachbedingung: Die Datenbank hat weniger Einträge als ihre maximale Kapazität. Die ältesten Einträge wurden gelöscht.\\
		Auslösendes Ereignis: Die definierte maximale Kapazität der Datenbank ist erreicht. 

		\renewcommand{\theenumi}{/FA\ifnum \value{enumi}<10 0\fi\arabic{enumi}0W/}
		\renewcommand{\labelenumi}{\theenumi}
		\renewcommand{\theenumii}{\arabic{enumii}}
		\renewcommand{\labelenumii}{/FA\ifnum \value{enumi}<10 0\fi\arabic{enumi}\arabic{enumii}W/}

		\item{\textbf{Smartphone Utilization}} \label{smartphoneutil} \\
		Vorbedingung: Es existiert eine Verbindung mit einem Endgerät. Die Sensoren des Endgeräts sind aktiviert. \\ Nachbedingung: Die Sensordaten vom Endgerät werden auf den Server übertragen. Die Daten werden auf dem Server gespeichert und von der Software bearbeitet. \\ Auslösendes Ereignis: Auf dem Endgerät sind neue Sensordaten verfügbar.

		\item{\textbf{Position tracken}}\\
		Ziel: Ein Bewegungsprofil erstellen, das mit den Sensorwerten in der Datenbank abgelegt wird.\\
		Vorbedingung: Mindestens ein Endgerät ist mit dem Server verbunden. Genau ein Nutzer ist als Fahrer identifiziert. Das Endgerät des Fahrers verfügt über Standortinformationen. Smartphone-Sensordaten können an den Server übermittelt werden. (~\ref{smartphoneutil})

		\renewcommand{\theenumi}{/FA\ifnum \value{enumi}<10 0\fi\arabic{enumi}0/}
		\renewcommand{\labelenumi}{\theenumi}
		\renewcommand{\theenumii}{\arabic{enumii}}
		\renewcommand{\labelenumii}{/FA\ifnum \value{enumi}<10 0\fi\arabic{enumi}\arabic{enumii}/}
		
		\item{\textbf{Remote Access erzeugen}} \label{remote} \\ Ziel: Der Benutzer kann die Software auf einem entfernten Endgerät benutzen.\\ Vorbedingungen: Das System läuft und hat eine Verbindung zum Internet.\\ Nachbedingung: Die Funktionen GUI-Konfiguration, Fahrinformation, Vergangene Daten, Karte und Einparkhilfe stehen auf dem Endsystem zur Verfügung.\\ Auslösendes Ereignis: Der Benutzer verbindet ein entferntes Endgerät über Eingabe einer URL im Browser mit dem Server.
		\setcounter{enumTemp}{\value{enumi}}
	\end{enumerate} 



	\begin{enumerate}

	\setcounter{enumi}{\value{enumTemp}}


		\item{\textbf{Fahrinformationen darstellen}} \label{liveData} \\ Ziel: Darstellung der aktuellen Sensorwerte auf dem Endgerät \\ Vorbedingung: Das System läuft und es werden durchgängig neue Daten zur Datenbank hinzugefügt. Es existiert eine Verbindung mit mindestens einem Endgerät. Dieses Gerät besitzt eine Konfiguration des GUI.\\ Nachbedingung: Das Endgerät zeigt diese Daten entsprechend der aktuellen GUI-Konfiguration an. Die Datenbank bleibt unverändert. \\ Auslösendes Ereignis: Das Endgerät empfängt Sensordaten vom Server.

		\item{\textbf{Aggregierte Funktionen berechnen}} \\ Ziel: Berechnung der aggregierten Funktionen. \\ Vorbedingung: Fahrinformation nach ~\ref{liveData} ist vorhanden. Gewünschte aggregierte Funktionen sind ausgewählt.\\ Nachbedingungen: Das Endgerät zeigt die Daten im UI an.\\ Ablauf: Die Software berechnet die gewünschten aggregierten Funktionen. Sie schickt die Informationen an das Endgerät. Das Endgerät zeigt die berechneten Daten entsprechend der GUI-Konfiguration an. 
		\\ Auslösendes Ereignis: Es kommen zur Berechnung der aggregierten Funktionen benötigte Sensorwerte am Server an.\\
		Aggregierte Funktionen sind:
		\begin{itemize}
			\item Treibstoffverbrauch: Die Software berechnet über den Tankfüllstand und die Zeit den aktuellen Treibstoffverbrauch.
			\item Durchschnittsverbrauch: Die Software berechnet über den Tankfüllstand und die Zeit den Durchschnittsverbrauch über die letzten 100km
			\item Restkilometer: Die Software berechnet eine Schätzung der Restkilometer abhängig vom aktuellen Treibstoffverbrauch.
			\item Strecke: Die Software berechnet die zurückgelegte Strecke über die Geschwindigkeit und die Zeit.
			\item Durchschnittsgeschwindigkeit
		\end{itemize}


	\setcounter{enumTemp}{\value{enumi}}
	\end{enumerate}


	\section{Einparkhilfe}
	
	\begin{enumerate}
	\setcounter{enumi}{\value{enumTemp}}

	\item{\textbf{Einparken vereinfachen}} \\ Ziel: Vereinfachen des rückwärts, beziehungsweise rückwärts-seitwärts Einparkens durch Anzeige einer Rückfahrkamera und Einblenden einer Visualisierung der möglichen Fahrstrecke. \\
	Vorbedingung: Es ist eine Rückfahrkamera vorhanden. \\ Nachbedingung: Das Bild der Rückfahrkamera wird angezeigt und regelmäßig aktualisiert. Die Visualisierung der Fahrbahn wird korrekt dargestellt und regelmäßig aktualisiert.\\
	Auslösendes Ereignis: Der Benutzer fordert das System auf, die Kamera einzuschalten. 

	\item{\textbf{Ultraschallsensoren visualisieren}} \label{ultrasound} \\ Ziel: Die Distanzwerte der Ultraschallsensoren visualisieren. 		\\ Vorbedingung: Es sind Ultraschallsensoren vorhanden.\\
	Nachbedingung: Die korrekten Messwerte werden angezeigt.\\
	Auslösendes Ereignis: Der Benutzer möchte die Messwerte anzeigen lassen.

	\item{\textbf{Einparkhilfe beenden}} \label{quit:passist} \\ Ziel: Die Einparkhilfe beenden. \\ Vorbedingung: Auf dem Bildschirm wird die Einparkhilfe angezeigt. \\ Nachbedingung (Erfolg): Die Einparkhilfe wird nicht mehr angezeigt.\\
	Auslösendes Ereignis: Der Benutzer fordert das System auf, die Einparkhilfe zu beenden. Alternativ: Das Fahrzeug fährt schneller als 10 km/h vorwärts.

	\setcounter{enumTemp}{\value{enumi}}

	\end{enumerate}
	
	
	\section{Karte und POI}
	
	\begin{enumerate}
	\setcounter{enumi}{\value{enumTemp}}
	\item{\textbf{Karte anzeigen}}
		\\Ziel: Die Karte der Umgebung anzeigen
		\\Vorbedingung: Das Endgerät des Fahrers hat eine Verbindung zum Netzwerk. Weiterhin kennt es den aktuellen Standort.
		\\Nachbedingung: Die Karte der Umgebung wird angezeigt.
  	 	\\Auslösendes Ereignis: Der Benutzer fordert die Karte an.
  	\item {\textbf{Tankstellen auflisten}}
		\\Ziel: Die Tankstellen in der Nähe finden
		\\Vorbedingung: Das Endgerät des Fahrers hat eine Verbindung zur Netzwerk.
		Die Lokalisierungsinformationen stehen zur Verfügung.
		\\Nachbedingung: Eine Liste, auf der die Tankstellen in der Nähe mit
		jeweiligem Preis bzw. jeweiliger Entfernung stehen, wird erzeugt und angezeigt. Auf der
		Karte werden die Tankstellen in der Nähe gekennzeichnet. 
		Der angezeigte Preis ist der des ausgewählten Treibstoffs.
		\\Auslösendes Ereignis: Der Tankfüllstand sinkt unter den festgelegten Mindeststand und ein Benutzer öffnet die Karte..
  	
	\item{\textbf{Werkstätten anzeigen}} 
		\\Ziel: Das OBD-System meldet ein Warnsignal. Daraufhin werden die umliegenden Werkstätten auf der Karte angezeigt.
		\\Vorbedingung: Das Endgerät des Fahrers hat eine Verbindung zum Netzwerk. Weiterhin kennt es den aktuellen Standort.
		\\Nachbedingung: Eine Liste der Werkstätten in der Nähe mit jeweiliger Entfernung wird erzeugt und angezeigt. Auf der Karte werden die Werkstätten in der Nähe gekennzeichnet. 
	 	\\Auslösendes Ereignis: Es liegt ein kritischer Sensorwert vor (außer Tankfüllstand). Ein Benutzer öffnet die Karte.
  	
  		\renewcommand{\theenumi}{/FA\ifnum \value{enumi}<10 0\fi\arabic{enumi}0W/}
		\renewcommand{\labelenumi}{\theenumi}
		\renewcommand{\theenumii}{\arabic{enumii}}
		\renewcommand{\labelenumii}{/FA\ifnum \value{enumi}<10 0\fi\arabic{enumi}\arabic{enumii}W/}

  	\item {\textbf{Route zu POI berechnen}}
  		\\Ziel: Der Fahrer wird über die Route zur gewünschten Tankstelle oder Werkstatt informiert. 
  		\\Vorbedingung: Die POI in der Nähe sind angezeigt. Das Endgerät des Fahrers hat eine Verbindung zum Netzwerk.
  		\\Nachbedingung: Die Route zum gewünschten POI wird auf der Karte angezeigt.
   		\\Auslösendes Ereignis: Der Fahrer wählt einen POI aus.

   		\renewcommand{\theenumi}{/FA\ifnum \value{enumi}<10 0\fi\arabic{enumi}0/}
		\renewcommand{\labelenumi}{\theenumi}
		\renewcommand{\theenumii}{\arabic{enumii}}
		\renewcommand{\labelenumii}{/FA\ifnum \value{enumi}<10 0\fi\arabic{enumi}\arabic{enumii}/}
  


	\setcounter{enumTemp}{\value{enumi}}
	\end{enumerate}

\end{document}
	

