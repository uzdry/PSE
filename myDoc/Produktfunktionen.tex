\documentclass[pflichtenheft.tex]{subfiles}

\begin{document}

\chapter{Produktfunktionen}

\section{Allgemeine Funktionen}

\begin{enumerate}

	%GUI-Elemente einblenden:
	
	\item{\textbf{GUI-Instrumemte auswählen}}
	
	\begin{enumerate}
	\item{\textbf{Instrumente anzeigen:}} \\ Vorbedingung: Server läuft. Es sind Endgeräte mit ihm vebunden.\\ Auslösendes Ereignis: Der Nutzer wählt in den Einstellungen auf dem Endgerät ein Instrument aus.\\ Nachbedingung: Das Instrument wird auf dem Screen angezeigt und zeigt die aktuellen Daten an. Falls keine aktuellen Daten vorhanden sind, steht es in neutraler Position (0).
	\end{enumerate} 
	
	%Datenhaltung:
	
	\item{\textbf{Datenhaltung}}

	\begin{enumerate}
	
		\item{\textbf{Protokoll führen}} \\ Ziel: Es werden eingehende Daten gespeichert. \\ Vorbedingung: Server läuft. Bluetooth-Verbindung mit OBD2 Adapter ist hergestellt. \\ Auslösendes Ereignis: Von der Bluetooth-Schnittstelle treffen Daten ein. %%TODO: Check BT API Node.js server 
		Nachbedingung: Die Daten, die von der Bluetooth-Schnittstelle kommen, befinden sich in der Datenbank und kommen über das Netzwerk auf Endgeräten an.
		
		\item{\textbf{Vergangene Daten aufrufen}} \\ Ziel: Es werden Daten von vergangenen Zeitpunkten dargestellt. \\ Vorbedingung: Es wurde bereits eine Fahrt getätigt. Der Server läuft.\\ Auslösendes Ereignis: Der Nutzer fordert Daten aus der Vergangenheit an. \\ Nachbedingung: Die Daten werden auf dem Endgerät angezeigt.
	
		\item{\textbf{Maximale Kapazität einstellen}} \\ Ziel: Es soll eine neue maximale Kapazität der Datenbank gesetzt werden. \\ Vorbedingung: Der Server läuft. \\ Auslösendes Ereignis: Der Benutzer versucht, eine neue maximale Kapazität der Datenbank zu setzen. \\ Nachbedingung: Die neue gespeicherte maximale Kapazität der Datenbank ist der vom Nutzer eingegebene Wert.
		
		\item{\textbf{Daten automatisch löschen}} \\ Ziel: Um Überfüllung der Datenbank zu vermeiden werden Daten gelöscht. \\ Vorbedingung: Der Server läuft. \\ Auslösendes Ereignis: Die definierte maximale Kapazität der Datenbank ist erreicht. \\ Nachbedingung: Die Datenbank hat weniger Einträge als ihre maximale Kapazität. Die ältesten Einträge wurden gelöscht.
		
	\end{enumerate} 

	\item{\textbf{Remote Access erzeugen}} \\ Ziel: Der Benutzer kann die Software auf einem entfernten Endystem benutzen.\\ Vorbedingungen: Das System läuft und es kommen Daten über den OBD2-Adapter an. Das System hat eine Verbindung zum Internet. \\ Ablauf: Der Benutzer verbindet ein entferntes Endgerät über Eingabe einer URL im Browser mit dem Server.\\ Nachbedingung (Erfolg): Die Funktionen GUI-Konfiguration, Fahrinformation, Vergangene Daten, Karte und Einparkhilfe stehen auf dem Endsystem zur Verfügung. \\ Nachbedingung (Fehlschlag): Das Endsystem zeigt an, dass die Verbindung zum Server fehlgeschlagen ist.

	\item{\textbf{Fahrinformationen darstellen}} \\ Ziel: Darstellung der aktuellen Sensorwerte auf dem Endgerät \\ Vorbedingung: Das System läuft und es werden durchgängig neue Daten zur Datenbank hinzugefügt. Es existiert eine Verbindung mit mindestens einem Endgerät. Dieses Gerät besitzt eine Konfiguration des GUI. \\ Auslösendes Ereignis: Das Endgerät empfängt Sensordaten vom Server.\\ Nachbedingung: Das Endgerät zeigt diese Daten entsprechend der aktuellen GUI-Konfiguration an. Die Datenbank bleibt unverändert.

	\item{\textbf{Aggregierte Funktionen berechnen}} \\ Ziel: Berechnung der aggregierten Funktionen. \\ Vorbedingung: FA/Fahrinformationen funktioniert. Gewünschte aggregierte Funktionen sind angegeben. \\ Auslösendes Ereignis: Es kommen zur Berechnung der aggregierten Funktionen benötigte Sensorwerte am Server an. \\ Ablauf: Die Software berechnet die gewünschten aggregierten Funktionen. Sie schickt die Informationen an das Endgerät. Das Endgerät zeigt die berechneten Daten entsprechend der GUI-Konfiguration an. \\ Nachbedingungen: Das Endgerät zeigt die Daten im UI an.  \\ Aggregierte Funktionen sind:
		\begin{itemize}
			\item Treibstoffverbrauch: Die Software berechnet über den Tankfüllstand und die Zeit den aktuellen Treibstoffverbrauch.
			\item Durchschnittsverbrauch: Die Software berechnet über den Tankfüllstand und die Zeit den Durchschnittsverbrauch über die letzten 100km
			\item Restkilometer: Die Software berechnet eine Schätzung der Restkilometer abhängig vom aktuellen Treibstoffverbrauch.
			\item Strecke: Die Software berechnet die zurückgelegte Strecke, über die Geschwindigkeit und die Zeit.
			\item Durchschnittsdauer/strecke: Die Software berechnet die durchschnittliche Strecke einer Fahrt, entweder über die Zeit oder über die Strecke.
		\end{itemize}

	\item{\textbf{Smartphone Utilization}}\\ Vorbedingung: Es existiert eine Verbindung mit einem Endgerät. Die Sensoren des Endgeräts sind aktiviert.\\ Auslösendes Ereignis: Auf dem Endgerät sind neue Sensordaten verfügbar. \\ Nachbedingung (Erfolg): Die Sensordaten vom Endgerät werden auf den Server übertragen. Die Daten werden auf dem Server gespeichert und von Software bearbeitet.\\  Nachbedingung (Fehlschlag): Die Sensoraten werden nicht übertragen. \\

	\item{\textbf{Einparkhilfe}}
	
	\begin{enumerate}

	\item{\textbf{Einparken vereinfachen}} \\ Ziel: Vereinfachen des rückwärts beziehungsweise rückwärts-Seitwärts Einparkens durch Anzeige einer Rückfahrkamera und Einblenden einer Visualisierung der möglichen Fahrstrecke. \\ Vorbedingung: Server läuft. \\ Auslösendes Ereignis:  Der Fahrer schaltet ins Rückwärtsgang. Alternativ: Der Benutzer fordert das System auf die Kamera einzuschalten. \\ Nachbedingung (Erfolg): Das Bild der Rückfahrkamera wird angezeigt und regelmäßig aktualisiert. Die Visualisierung der Fahrbahn wird korrekt dargestellt und regelmäßig aktualisiert. \\ Nachbedingung (Fehlschlag): Das Bild der Rückfahrkamera wird nicht angezeigt, ist fehlerbehaftet oder wird nicht aktualisiert. Die Visualisierung der Fahrbahn wird nicht dargestellt, ist nicht korrekt oder wird nicht aktualisiert. 

	\item{\textbf{Ultraschallsensoren visualisieren}} \\ Ziel: Die Distanzwerte der Ultraschallsensoren visualisieren. \\ Vorbedingung: Eine Kamera ist vorhanden und eingeschaltet. Der Server läuft. \\ Auslösendes Ereignis: Der Fahrer schaltet in den Rückwärtsgang. Alternativ: Der Benutzer möchte die Messwerte anzeigen.\\ Nachbedingung (Erfolg): Die korrekten Messwerte werden angezeigt. \\ Nachbedingung (Fehlschlag): Es werden keine oder falsche Messwerte angezeigt.

	\item{\textbf{Einparkhilfe beenden}} \\ Ziel: Das System soll die Darstellung der Einparkhilfe automatisch vom Bildschirm entfernen, wenn die Daten nicht mehr sinnvoll dargestellt werden können. \\ Vorbedingung: Auf dem Bildschirm wird die Einparkhilfe angezeigt. \\ Nachbedingung (Erfolg): Die Einparkhilfe wird nicht mehr angezeigt. \\ Auslösendes Ereignis: Der Benutzer fordert das System auf, die Einparkhilfe zu beenden. Alternativ: Das Fahrzeug fährt schneller als 5 km/h vorwärts. \\  Nachbedingung (Fehlschlag): Die Einparkhilfe verschwindet nicht von dem Bildschirm.

	\end{enumerate}
	
	
	\item{\textbf{Karte und POI}}
	
	\begin{enumerate}
	
	\item{\textbf{Karte anzeigen}}
	\begin{enumerate}
		\item 
		Vorbedingung: Die Endgeräte vom Fahrer haben eine Verbindung zur Netzwerk. Die
		Lokalisierungsinformationen stehen zur Verfügung.
		\\Auslösendes Ereignis: Der Benutzer fordert die Karte an.
	 	\\Nachbedingung:Eine Anfrage nach der Karte in Umgebung mit
	 	Lokalisierungsinformationen wird am Drittanbietern geschickt, der die
	 	Kartendienst anbieten kann.
	 	\item
		Vorbedingung: Eine Anfrage nach der Karte in Umgebung mit
		Lokalisierungsinformationen ist am Kartendienstanbietern geschickt. Die
		Endgeräte vom Fahrer haben eine Verbindung zur Netzwerk.
		\\Auslösundes Ereignis: Der Drittanbieter antwortet die Anfrage.
		\\Nachbedingung: Die Karte von Umgebung wird angezeigt.
  	\end{enumerate}
  	
  	\item {\textbf{Tankstellen auflisten}}
  	\begin{enumerate} 
  		\item 
		Vorbedingung: Die Endgeräte vom Fahrer haben eine Verbindung zur Netzwerk. Die
		Lokalisierungsinformationen stehen zur Verfügung.
		\\Auslösendes Ereignis: Das OBD-system detektiert einen niedrigen
		Tankfüllstand
		\\Nachbedingung:Eine Anfrage nach den Tankstellen in der Nähe mit
	 	Lokalisierungsinformationen wird am Drittanbietern geschickt, der die
	 	Kartendienst anbieten kann.
	 	\item
		Vorbedingung: Eine Anfrage nach den Tankstellen in der Nähe ist am
		Kartendienstanbietern geschickt. Die Endgeräte vom Fahrer haben eine
		Verbindung zur Netzwerk.
		\\Auslösundes Ereignis: Der Drittanbieter antwortet die Anfrage.
		\\Nachbedingung: Eine Liste, auf der die Tankstellen in der Nähe mit
		jeweiligen Preis bzw. Entfernung stehen, werden erzeugt und angezeigt. Auf der
		Karte werden die Tankstellen in der Nähe kenngezeichnet.
  	\end{enumerate}
  	
  	\item {\textbf{Route zur gewünschten Tankstelle berechnen}} 
  	\begin{enumerate}
  		\item 
  		Vorbedingung: Die Tankstellen in der Nähe sind angezeigt. Die Endgeräte vom
  		Fahrer haben die Verbindung zur Netzwerk.
  		\\Auslösendes Ereignis: Der Fahrer wählt eine Tankstelle aus.
  		\\Nachbedingung: Eine Anfrage nach der Route zur vom Fahrer ausgewählten
  		Tankstelle wird am Kantendienstanbietern geschickt.
  		\item
  		Vorbedingung: Eine Anfrage nach der Route zur Tankstelle ist am
  		Drittanbieter von Kartendienst geschickt. Die Endgeräte vom Fahrer haben die
  		Verbindung zur Netzwerk.
  		\\Auslösendes Ereignis: Der Drittanbieter antwortet die Anfrge.
  		\\Nachbedingung: Die Route zur gewünschten Tankstelle wird auf Karte
  		angezeigt.
  		\end{enumerate}
  
	\item{\textbf{Werkstätten anzeigen}} 
	\begin{enumerate}
		\item 
		Vorbedingung: Die Endgeräte vom Fahrer haben eine Verbindung zur Netzwerk.
		Die Lokalisierungsinformationen stehen zur Verfügung.
		\\Auslösendes Ereignis: Das OBD-system detektiert einen kritischen Sensorwert.
		\\Nachbedingung:Eine Anfrage nach der nächsten Werkstatt mit
	 	Lokalisierungsinformationen wird am Drittanbietern geschickt, der die
	 	Kartendienst anbieten kann.
	 	\item
		Vorbedingung: Eine Anfrage nach der Route zur nächsten Werkstatt ist am
		Kartendienstanbietern geschickt. Die Endgeräte vom Fahrer haben eine
		Verbindung zur Netzwerk.
		\\Auslösundes Ereignis: Der Drittanbieter antwortet die Anfrage.
		\\Nachbedingung: Auf der Karte wird die Route zur nächsten Werkstatt
		angezeigt.
		\end{enumerate}

	\end{enumerate}
\end{enumerate}


%noch nicht verarbeitetes:

\subsection{\mkfa Statistiken} Der Benutzer kann sich Statistiken der gespeicherten Sensorwerte anzeigen lassen.

\subsection{\mkfa Erweiterbarkeit} Drittentwickler können Module für die Verarbeitung und Auswertung weiterer Sensordaren schreiben.

\subsubsection{\mkfaw Fahrer} Ein Nutzer, der mit dem Server verbunden ist, kann sich mit der Software auf seinem Endgerät als Fahrer identifizieren (/FA0070/, /FA1020W/).
\end{document}
	
	

