\documentclass[pflichtenheft.tex]{subfiles}
\usepackage{tabularx}
\begin{document}

\chapter{Qualitätsbestimmung}

\begin{table}
\centering
\begin{tabular}{ | l | l | l | l | l | }
    \hline
    \textbf{Produktqualität} & \textbf{Sehr gut} & \textbf{Gut} & \textbf{Normal} & 
    	\textbf{Nicht relevant}\\ \hline
    &&&& \\
    \Large{Funktionalität}	&  &  &  &  \\    \hline
    Angemessenheit 			&  &  &  &  \\    \hline
    Richtigkeit    			&  &  &  &  \\    \hline
    Interoperabilität   	&  &  &  &  \\    \hline
    Ordnungsmäßigkeit		&  &  &  &  \\    \hline
    Sicherheit				&  &  &  &  \\    \hline
    &&&& \\
    \Large{Zuverlässigkeit}  &  &  &  &  \\    \hline
    Reife					&  &  &  &  \\    \hline
    Fehlertoleranz			&  &  &  &  \\    \hline
    Wiederherstellbarkeit	&  &  &  &  \\    \hline
    &&&& \\
    \Large{Benutzbarkeit}	&  &  &  &  \\    \hline
    Verständlichkeit		&  &  &  &  \\    \hline
    Erlernbarkeit			&  &  &  &  \\    \hline
    Bedienbarkeit			&  &  &  &  \\    \hline
    &&&& \\
    \Large{Effizienz}		&  &  &  &  \\    \hline
    Zeitverhalten			&  &  &  &  \\    \hline
    Verbrauchsverhalten		&  &  &  &  \\    \hline
    &&&& \\
    \Large{Änderbarkeit}	&  &  &  &  \\    \hline
    Analysierbarkeit		&  &  &  &  \\    \hline
    Modifizierbarkeit		&  &  &  &  \\    \hline
    Stabilität				&  &  &  &  \\    \hline
    Prüfbarkeit				&  &  &  &  \\    \hline
    &&&& \\
    \Large{Übertragbarkeit}	&  &  &  &  \\    \hline
    Installierbarkeit		&  &  &  &  \\    \hline
    Konformität				&  &  &  &  \\    \hline
    Austauschbarkeit		&  &  &  &  \\    \hline
    
    
\end{tabular}
\end{table}

\end{document}